Many fundamental biological tasks require unsupervised learning where ground truth labels are unavailable, but shallow unsupervised machine learning methods have poor performance on these tasks due to the complexity of the problem. Deep learning models, with their strong representation power, have been widely applied to solve challenging tasks; however, they usually require large amounts of labeled data. To take advantage of the strong representation power of deep learning while applying them to unsupervised tasks, we developed several hybrid models that combine deep neural networks and unsupervised machine learning models. We used these models to improve performance on unsupervised biological tasks, including cell type clustering, basecalling, and lead optimization.
First, we present an unsupervised cell type clustering model for recently developed  single-molecule spatially resolved transcriptomics data, where a deep Neural Network (NN) encoder is used to generate low-dimensional Gaussian distributed gene embedding, so it can be combined with the spatial relationship using a Gaussian-Multinomial Mixture Model developed by us to predict the cell-type clustering. The second problem we try to tackle is to call m6A methylated bases in RNA generated from long-read sequencing. m6A modification plays essential roles in regulating gene expression while lacking an efficient way to detect it systemically. The long-read sequencing from Oxford Nanopore Technologies has been shown to be sensitive to post-transcriptional modification, but an m6A sensitive basecaller for directly detecting this subtle sequencing signal has not yet been developed. We used a CNN-RNN (Convolutional-Recurrent Neural network) model previously developed by us for canonical basecalling to train a Non-homogeneous HMM (NHMM) where its transition matrix is conditioned on the deep NN output. Using the hybrid synthetically m6A methylation data sampled from the NHMM, we were able to train a NN basecaller to call m6A base. We applied our method to call the methylome on Yeast RNA without requiring knock-out comparison data. For the third application, I propose a deep generative model with a deep Graph Neural Network and diffusion model to lead optimization problems in drug discovery, where the binding affinity is unreachable, and the deep learning model is suitable to deal with the complexity of the problem.
